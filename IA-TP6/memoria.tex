\documentclass[a4paper, 10pt]{report}
\usepackage[spanish]{babel}
\usepackage[utf8]{inputenc}

\usepackage{parskip}
\usepackage[hidelinks]{hyperref}
\usepackage{fancyvrb}
\usepackage{graphicx}

\usepackage{multicol}
\setlength{\columnsep}{1.5cm}
\setlength{\columnseprule}{0.2pt}

\addto\captionsspanish{
  \renewcommand{\tablename}{Tabla}
}
\usepackage[margin=1.2in,headheight=13.6pt]{geometry}

\usepackage{fancyhdr}
\fancyhf{}
\pagestyle{fancy}
\lhead{\emph{Inteligencia Artificial 2019-20}}
\rhead{\emph{Memoria Trabajo TP6-1}}

\usepackage{float}

\usepackage{xcolor}

\addto\captionsspanish{\renewcommand{\chaptername}{Tarea}}

\usepackage{subfiles}

\begin{document}

\begin{titlepage} % Suppresses headers and footers on the title page
	\centering % Centre everything on the title page	

	
	\vspace*{\baselineskip} % White space at the top of the page
	
	%------------------------------------------------
	%	Title
	%------------------------------------------------
	
	\rule{\textwidth}{1.6pt}\vspace*{-\baselineskip}\vspace*{2pt} % Thick horizontal rule
	\rule{\textwidth}{0.4pt} % Thin horizontal rule
	
	\vspace{0.75\baselineskip} % Whitespace above the title
	
	{\bfseries \color[rgb]{0.1608,0.2392,0.4214}\Huge MEMORIA TP6\,-1 \\ 
	\huge Búsqueda local y Propagación de restricciones\\} % Title
	
	\vspace{0.75\baselineskip} % Whitespace below the title
	
	\rule{\textwidth}{0.4pt}\vspace*{-\baselineskip}\vspace{3.2pt} % Thin horizontal rule
	\rule{\textwidth}{1.6pt} % Thick horizontal rule
	
	\vspace{2\baselineskip} % Whitespace after the title block
	
	%------------------------------------------------
	%	Subtitle
	%------------------------------------------------
	{\scshape \LARGE Inteligencia Artificial}
	
	\vspace*{0.5\baselineskip} % Whitespace under the subtitle
	
	{\large Curso 2019-2020} % Subtitle or further description

	
	\vspace*{4\baselineskip} % Whitespace under the subtitle
	
	%------------------------------------------------
	%	Editor(s)
	%------------------------------------------------
	
	{\scshape Autor}
	
	\vspace{0.5\baselineskip} % Whitespace before the editors
	
	{\scshape\ \Large Fernando Peña Bes} % Editor list
	
	\vspace{0.5\baselineskip} % Whitespace below the editor list
	
	{\scshape \large nia: 756012} % Editor list
	
	\vspace{0\baselineskip} % Whitespace below the editor list
	
	\rule{0.2\textwidth}{0.4pt} % Thin horizontal rule

	
	\textit{\large Grado en Ingeniería Informática} % Editor affiliation
	
	\vspace{0.01\baselineskip} % Whitespace before the editors
	
 % Editor affiliation
	
	\vfill % Whitespace between editor names and publisher logo
	
	%------------------------------------------------
	%	Publisher
	%------------------------------------------------
	\includegraphics[scale=0.8]{eina}
	\vspace{0.5\baselineskip}

		
	{\large \today} % Publication year
	
\end{titlepage}

\tableofcontents




\chapter*{Introducción}
\addcontentsline{toc}{chapter}{Introducción} \markboth{INTRODUCTION}{}

En este documento se trata el trabajo realizado en el trabajo práctico TP6 de la asignatura Inteligencia Artificial de la Universidad de Zaragoza.

El trabajo consta de tres tareas. En la primera se resuelven problemas de \textbf{búsqueda local} y optimización y aplicándolo al problema de las \textbf{8-reinas}. En la segunda, se resuelve el problema de resolución de \textbf{sudokus} mediante \textbf{propagación de restricciones}, y por último, en la tercera se combina la idea de búsqueda local con la propagación de restricciones usando el algoritmo \textbf{\texttt{min-conflicts}}, aplicándolo también al problema de las 8-reinas.

Para realizar la práctica se ha usado el entorno de desarrollo Eclipse para Java y el código perteneciente al libro \textit{Artificial Intelligence: A Modern Approach} \cite{AIMA}, en la versión 1.8 (se puede descargar en \url{https://github.com/aimacode/aima-java/releases/tag/aima3e-v1.8.1}).

% Mostrar los resultados y comentarios de los experimentos realizados en esta primera parte del trabajo.

\subfile{secciones/tarea1}





\section{Conclusiones}
En esta práctica se han evaluado diferentes algoritmos y heurísticas y se ha visto cuales son más adecuadas para resolver el problema del 8-puzzle. El mejor algoritmo de los evaluados, es el A* con la heurística de Manhattan.
\chapter{Resolución de sudokus mediante propagación de restricciones  y búsqueda}

\chapter{Propagación de restricciones y búsqueda local}

\chapter*{Conclusiones}
\addcontentsline{toc}{chapter}{Conclusiones} \markboth{CONCLUSIONES}{}

\begin{thebibliography}{2}
\addcontentsline{toc}{chapter}{Bibliografía} \markboth{BIBLIOGRAFIA}{}
      \bibitem{AIMA}
      Stuart Rusell and Peter Norvig,
      \textit{Artificial Intelligence: A Modern Approach}.

      \bibitem{}
      \textit{Apuntes de la asignatura Inteligencia Artificial}, Curso 2019-20.
      
\end{thebibliography}

\end{document}